%---------------ПРЕАМБУЛА---------------------

\documentclass[a4paper,14pt]{article}

%----------Работа с русским языком------------
\usepackage[warn]{mathtext}	
\usepackage{cmap}					% поиск в PDF
\usepackage[T2A]{fontenc}			% кодировка
\usepackage[utf8]{inputenc}			% кодировка исходного текста
\usepackage[english,russian]{babel}	% локализация и переносы

%------Дополнительная работа с математикой----
\usepackage{amsmath,amsfonts,amssymb,amsthm,mathtools} % AMS
\usepackage{icomma}                 % "Умная" запятая: $0,2$ --- число, $0, 2$ --- перечисление

%-------------Номера формул-------------------
%\mathtoolsset{showonlyrefs=true} % Показывать номера только у тех формул, на которые есть \eqref{} в тексте.
%\usepackage{leqno} % Немуреация формул слева

%-------------Свои команды--------------------
\DeclareMathOperator{\sgn}{\mathop{sgn}}

%--Перенос знаков в формулах (по Львовскому)--
\newcommand*{\hm}[1]{#1\nobreak\discretionary{}
	{\hbox{$\mathsurround=0pt #1$}}{}}

%----------Работа с картинками----------------
\usepackage{graphicx}  % Для вставки рисунков
\graphicspath{{images/}{images2/}}  % папки с картинками
\setlength\fboxsep{3pt} % Отступ рамки \fbox{} от рисунка
\setlength\fboxrule{1pt} % Толщина линий рамки \fbox{}
\usepackage{wrapfig} % Обтекание рисунков текстом

%----------Работа с таблицами-----------------
\usepackage{array,tabularx,tabulary,booktabs} % Дополнительная работа с таблицами
\usepackage{longtable}  % Длинные таблицы
\usepackage{multirow} % Слияние строк в таблице

%--------------Теоремы------------------------
\theoremstyle{plain} % Это стиль по умолчанию, его можно не переопределять.
\newtheorem{theorem}{Теорема}[section]
\newtheorem{proposition}[theorem]{Утверждение}

\theoremstyle{definition} % "Определение"
\newtheorem{corollary}{Следствие}[theorem]
\newtheorem{problem}{Задача}[section]

\theoremstyle{remark} % "Примечание"
\newtheorem*{nonum}{Решение}

%---------Программирование--------------------
\usepackage{etoolbox} % логические операторы

%-------------Страница------------------------
\usepackage{extsizes} % Возможность сделать 14-й шрифт
\usepackage{geometry} % Простой способ задавать поля
\geometry{top=25mm}
\geometry{bottom=25mm}
\geometry{left=30mm}
\geometry{right=30mm}

\usepackage{fancyhdr} % Колонтитулы
\pagestyle{fancy}
\fancyhead{}
\fancyfoot{}
\fancyhead[RO]{\normalsize \textit \thepage} %номер страницы справа на нечетных
\fancyhead[CO]{\normalsize \textit{\S 1.1. Релятивисткая волновая механика}}
%\renewcommand{\headrulewidth}{0,1mm}  % Толщина линейки, отчеркивающей верхний колонтитул
%\lfoot{Нижний левый}
%\rfoot{Нижний правый}
%\rhead{Верхний правый}
%\chead{Верхний в центре}
%\lhead{Верхний левый}
% \cfoot{Нижний в центре} % По умолчанию здесь номер страницы

\usepackage{setspace} % Интерлиньяж
%\onehalfspacing % Интерлиньяж 1.5
%\doublespacing % Интерлиньяж 2
%\singlespacing % Интерлиньяж 1

\usepackage{lastpage} % Узнать, сколько всего страниц в документе.

\usepackage{soul} % Модификаторы начертания

\usepackage{hyperref}
\usepackage[usenames,dvipsnames,svgnames,table,rgb]{xcolor}
\hypersetup{				% Гиперссылки
	unicode=true,           % русские буквы в раздела PDF
	pdftitle={Заголовок},   % Заголовок
	pdfauthor={Автор},      % Автор
	pdfsubject={Тема},      % Тема
	pdfcreator={Создатель}, % Создатель
	pdfproducer={Производитель}, % Производитель
	pdfkeywords={keyword1} {key2} {key3}, % Ключевые слова
	colorlinks=true,       	% false: ссылки в рамках; true: цветные ссылки
	linkcolor=blue,          % внутренние ссылки
	citecolor=green,        % на библиографию
	filecolor=magenta,      % на файлы
	urlcolor=cyan           % на URL
}

%----------------Цвета------------------------
\usepackage{color,colortbl}
\definecolor{O}{RGB}{154,175,255}
\definecolor{B}{RGB}{202,215,255}
\definecolor{F}{RGB}{255,244,234}
\definecolor{K}{RGB}{255,196,111}
\definecolor{G}{RGB}{255,242,161}
\definecolor{M}{RGB}{255,96,96}
%\renewcommand{\familydefault}{\sfdefault} % Начертание шрифта

%----------------Прочее-----------------------
\usepackage{lipsum}
\usepackage{multicol} % Несколько колонок
\usepackage{tikz} % Работа с графикой
\usepackage{pgfplots}
\usepackage{pgfplotstable}
%\usepackage{floatrow}
\usepackage{caption}
\usepackage[caption2]{ccaption}
\DeclareCaptionLabelSeparator{dot}{. }
\captionsetup{justification=centering,labelsep=dot}
\usepackage{wasysym}
\setlength{\fboxrule}{0.5pt}
\usepackage{mathrsfs}
\usepackage{makeidx}
\makeindex
%---------------------------------------------
%---------------------------------------------
%---------------ДОКУМЕНТ----------------------
%---------------------------------------------
%---------------------------------------------
\begin{document}
%------------Команды и окружения--------------
\newcommand{\anonsection}[1]{\section*{#1}\addcontentsline{toc}{part}{#1}}
\newcommand{\un}{\underline}
\newcommand{\ve}{\vec}
%---------------------------------------------

%------------------ТЕКСТ----------------------
опубликовать свое релятивистсское волновое уравнение, оно уже было независимо переоткрыто Оскаром Клейном [7] и Вальтером Гордоном [8]. По этой причине релятивистский вариант называется "уравнение Клейна-Гордона".

Шредингер вывел свое релятивистское волновое уравнение, заметив, что гамильтониан $H$ и импульс $\mathbf{p}$ "электрона Лоренца" c массой $m$ и зарядом $e$, находящегося во внешнем векторном потенциале $\mathbf{A}$ и кулоновском потенциале $\phi$, связаны следующим соотношением \footnote[1]{Это соотношение лоренц-инвариантно, поскольку величины $\mathbf{A}$ и $\phi$ при преобразованиях Лоренца изменяются точно так же, как и $\mathbf{p}$ и \textit{E} соответственно. Гамильтониан \textit{H} и импульс $\mathbf{p}$ Шрёдингер представлял в виде частных производных действия, однако это неважно для нашего рассмотрения.}
\begin{align}\label{eq1}
0=(H+e\phi)^{2}-c^{2}(\mathbf{p}+e \mathbf{A} / c)^{2}-m^{2} c^{4}
\end{align}
Соотношения де Бройля $(1. 1.1)$ для \textit{свободной} частицы, представленной плоской волной $\exp \{2 \pi i(\boldsymbol{\kappa} \cdot \mathbf{x}-\nu t)\}$, можно получить, если произвести отождествление
\begin{align}\label{eq2}
\mathbf{p}=h \mathbf{k} \rightarrow-i \hbar \nabla, \quad E=h \nu \rightarrow i \hbar \frac{\partial}{\partial t}
\end{align}
где $\hbar$ -- удобное обозначение (введенное Дираком) для $h / 2 \pi .$ Исходя из чисто формальной аналогии, Шредингер предположил, что электрон во внешних полях $\mathbf{A}, \;\phi$ должен описываться волновой функцией $\psi(\mathbf{x}, t)$, удовлетворяющей уравнению, получаемому при помощи той же самой замены в \ref{eq1}
\begin{align}\label{eq3}
0=\left[\left(i \hbar \frac{\partial}{\partial t}+e \phi\right)^{2}-c^{2}\left(-i \hbar \nabla+\frac{e \mathbf{A}}{c}\right)^{2}-m^{2} c^{4}\right] \psi(\mathbf{x}, t)
\end{align}
В частности, для стационарных состояний в атоме водорода справедливы равенства $\mathbf{A}=0$ и $\phi=e /(4 \pi r) .$ Кроме того, в этом случае $\psi$ зависит от времени $t$ экспоненциально: $\exp (-i E t / \hbar) .$ Поэтому \ref{eq3} сводится к уравнению
\begin{align}\label{eq4}
0=\left[\left(E+\frac{e^{2}}{4 \pi r}\right)^{2}-c^{2} \hbar^{2} \nabla^{2}-m^{2} c^{4}\right] \psi(\mathbf{x})
\end{align}
Решения уравнения \ref{eq4} с наложенными на них разумными граничными условиями, определяют уровни энергии [9]
\begin{align}\label{eq5}
E=m c^{2}\left[1-\frac{\alpha^{2}}{2 n^{2}}-\frac{\alpha^{4}}{2 n^{4}}\left(\frac{n}{l+1 / 2}-\frac{3}{4}\right)+\ldots\right]
\end{align}
\end{document}
