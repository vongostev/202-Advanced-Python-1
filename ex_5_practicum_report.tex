\documentclass{article}
\usepackage[T2A]{fontenc}
\usepackage[utf8]{inputenc}
\usepackage[russian, english]{babel}
\usepackage{float}
\usepackage{amsmath}
\usepackage{physics}
\usepackage{fancyhdr}
\begin{document}
	\Russian
	\begin{center}
		ИЗМЕРЕНИЕ УДЕЛЬНОГО ЗАРЯДА ЭЛЕКТРОНА
	\end{center}
	В ходе данной лаборатоной работы необходимо было получить удельный заряд электрона $e$/$m$. Есть два метода получения данной константы по закону "Трех вторых" и с помощью метода магнетрона. 
	
	Закон "трех вторых" устанавливает связь силы тока и напряжения $I \sim U^{3/2}$, при малых областях тока. Проводя все необходимые выкладки устанавливается формула для нахождения удельного заряда электрона.
	\begin{equation}
		\frac{e}{m} = \biggl(\frac{C}{A*\varepsilon}\biggl)^2
	\end{equation}
	где $C \equiv I=CU^{3/2}$, A = 2.85, $\varepsilon$ - электрическая постоянная
	
	В ходе обработки были полученные следующие значения С $= 0.0115 \frac{мA}{B^{3/2}}$ и $e$/$m$ = $1.2*10^{12}\frac{\text{Кл}}{\text{кг}}$
	
	Значение $e$/$m$ может быть найдено по траектории электрона в заданном магнитном. Это называется методом магнетрона. Из графика силы тока от магнитной индукции мы должны зафиксировать спад силы тока и получить значение магнитной индукции. С помощью формулы:
	\begin{equation}
		\frac{e}{m} = \frac{8U}{B^2r^2}
	\end{equation}
	где $U$ - напряжение, $B$ - магнитная индукция при спаде тока, $r$ - радиус анода
	
	Был получен уделный заряд электрона со снятых данных при 3 напряжений:
	
	$e$/$m$ = $9.6*10^{11}\frac{\text{Кл}}{\text{кг}}$ (U = 30B)
	
	$e$/$m$ = $2.6*10^{12}\frac{\text{Кл}}{\text{кг}}$ (U = 40B)
	
	$e$/$m$ = $4.5*10^{12}\frac{\text{Кл}}{\text{кг}}$ (U = 50B)
	
	Истинное значение $e$/$m$ = $1.75*10^{11}\frac{\text{Кл}}{\text{кг}}$
	
	 
\end{document}