%Grigoriev Semyon
\documentclass[a5paper,8pt]{article}
\usepackage[9pt]{extsizes}
\usepackage[utf8]{inputenc}
\usepackage[T2A]{fontenc}
\usepackage{geometry}
\geometry{a5paper,right=20mm, left=20mm, tmargin=20mm, bmargin=20mm,headsep=4mm}
\usepackage{float}
\usepackage{amsmath}
\usepackage{babel}

\setcounter{page}{19}
\usepackage{fancyhdr}
\fancyhead[R]{\thepage}
\fancyhead[L]{}
\fancyhead[C]{\textit{\textsection 1.1 Релятивисткая волновая механика}}
\fancyfoot[L]{2*}
\fancyfoot[C]{}

\begin{document}
\sloppy
\pagestyle{fancy}
\setlength{\abovedisplayskip}{6pt}
\setlength{\abovedisplayshortskip}{6pt}
\setlength{\belowdisplayskip}{6pt}
\setlength{\belowdisplayshortskip}{6pt}
\noindent
опубликовать свое релятивисткое волновое уравнение, оно было уже независимо переотокрыто Оскаром Клейном [7] и Вальтером Гордоном [8]. По этой причине релятивисткий вариант называется ''уравнением Клейна-Гордона''. \newline \indent Шредингер вывел свое релятивистское волновое уравнение, заметив, что гамильтониан $H$ и импульс $\textbf{p}$ ''электрона Лоренца'' с массой $m$ и зарядом $e$, находящегося во внешнем векторном потенциале $\textbf{A}$ и кулоновском потенциале $\phi$, связаны следующим соотношением\footnote{)\indentЭто соотношение лоренц-инвариантно, поскольку величины $\textbf{A}$ и $\phi$ при преобразованиях Лоренца изменяются точно так же, как $c\textbf{p}$ и $E$ соответственно. Гамильтониан $H$ и импульс $\textbf{p}$ Шредингер представлял в виде частных производных действия, однако это неважно для нашего рассмотрения.}$^)$:
\begin{equation}
\label{formula1}
0 = (H+e\phi )^2-c^2(\textbf{p} +e\textbf{A} /c)^2-m^2c^4.
\end{equation}
\noindent
Соотношения де Бройля для \textit{cвободной} частицы, представленной плоской волной $\exp\{2 \pi i(\boldsymbol{\kappa}\cdot\textbf{x}-vt)\} $, можно получить, если произвести отождествление
\begin{equation}
\textbf{p} = h\textbf{k}\rightarrow -i\hbar\nabla,\indent E=h\nu \rightarrow i\hbar \frac{\partial}{\partial t},
\end{equation}
\noindent
где $\hbar$ — удобное обозначение (введенное Дираком) для $h/2 \pi$. Исходя из чисто формальной аналогии Шредингер предположил, что электрон во внешних полях $\textbf{A}$, $\phi$ должен описываться волновой функцией $\psi(\textbf{x},t)$, удовлетворяющей уравнению, получаемому при помощи той же самой замены в (\ref{formula1}):
\begin{equation}
\label{formula2}
0 = \biggl[\biggl(i\hbar \frac{\partial}{\partial t}+e\phi \biggr)^2 - c^2\biggl(-i\hbar \nabla +\frac{e \textbf{A}}{c}\biggr)^2-m^2c^4\biggr]\psi (\textbf {x},t).
\end{equation}
В частности, для стационарных состояний в атоме водорода справедливы равенства $\textbf{A} = 0$ и $\psi = e/(4\pi r)$. Кроме того, в этом случае $\psi$ зависит от времени $t$ экспоненциально: $\exp(-iEt/\hbar)$. Поэтому (\ref{formula2}) сводится к уравнению
\begin{equation}
\label{formula3}
0 = \biggl[\biggl(E+\frac{e^2}{4\pi r}\biggr)^2-c^2\hbar^2\nabla^2-m^2c^4\biggr]\psi (\textbf{x}).
\end{equation}
Решения уравнения (\ref{formula3}) с наложенными на них разумными граничными условиями, определяют уровни энергии [9]
\begin{equation}
E = mc^2\biggl[1-\frac{\alpha^2}{2n^2}-\frac{\alpha^4}{2n^4}\biggl(\frac{n}{l+1/2}-\frac{3}{4}\biggl)+\ldots \biggr],
\end{equation}
где $\alpha\equiv e^2/(4\pi\hbar c)$ -- ''постоянная тонкой структуры'', численное значение которой составляет приблизительно $1/137$, $n$ -- положительное целое число, а $l$ -- орбитальный угловой момент в единицах $\hbar$, принимающий целочисленные значения в интервале $0\le l\le n-1$. Наличие...
\end{document}