\documentclass[a5paper,8pt]{article}
\usepackage[utf8]{inputenc}
\usepackage[T2A]{fontenc}
\usepackage{geometry}
\geometry{tmargin=2cm, bmargin=2cm, lmargin=1.7cm, rmargin=1.7cm}
\usepackage{float}
\usepackage{amsmath}
\usepackage{physics}
\usepackage{graphicx}
\usepackage[russian, english]{babel}

\pagestyle{myheadings}
\setcounter{page}{19}

\begin{document}

Шредингер вывел свое релятивистское волновое уравнение, заметив, что гамильтониан $H$ и импульс $\textbf{p}$ \textquotedblleftэлектрона Лоренца\textquotedblright\ с массой $m$ и зарядом $e$, находящегося во внешнем векторном потенциале $\vb{A}$ и кулоновском потенциале $\phi$, связаны следующим соотношением:
\begin{equation}
    0 = (H + e \phi)^2 - c^2 (\textbf{p} + e \textbf{A}/c)^2 - m^2 c^4. \label{eq:1.1.2}
\end{equation}
Соотношения де Бройля \eqref{eq:1.1.1} для \textit{свободной} частицы, представленной плоской волной $\exp \{ 2 \pi i (\boldsymbol{\kappa} \cdot \textbf{x} - \nu t) \}$, можно получить, если произвести отождествление
\begin{equation}
    \textbf{p} = h \textbf{k} \rightarrow -i \hbar \nabla, \;\;\;\;\;\;\; E = h \nu \rightarrow i \hbar \frac{\partial}{\partial t}, \label{eq:1.1.3}
\end{equation}
где $\hbar$ --- удобное обозначение (введенное Дираком) для $h / 2 \pi$. Исходя из чисто формальной аналогиии Шредингер предположил, что электрон во внешних полях $\textbf{A}, \phi$ должен описываться волновой функцией $\psi (x, t)$, удовлетворяющей уравнению, получаемому при помощи той же самой замены в \eqref{eq:1.1.2}:
\begin{equation}
    0 = \left[\left(i \hbar \frac{\partial}{\partial t} + e \phi \right)^2 - c^2 \left(- \hbar \nabla + \frac{e \textbf{A}}{c} \right)^2 - m^2 c^4\right] \psi(\textbf{x}, t). \label{eq:1.1.4}
\end{equation}
В частности, для стационарных состояний в атоме водорода справедливы равенства $\textbf{A} = 0$ и $\phi = e/(4 \pi r)$. Кроме того, в этом случае $\psi$ зависит от времени $t$ експоненциально: $\exp (-i E t / h)$. Поэтому \eqref{eq:1.1.4} сводится к уравнению
\begin{equation}
    0 = \left[\left(E + \frac{e^2}{4 \pi r} \right)^2 - c^2 \hbar^2 \nabla^2  - m^2 c^4\right] \psi(\textbf{x}). \label{eq:1.1.5}
\end{equation}
Решения уравнения \eqref{eq:1.1.5} с наложенными на них разумными граничными условиями, определяют уровни энергии \cite{book:9}
\begin{equation}
    E = mc^2 \left[ 1 - \frac{\alpha^2}{2n^2} - \frac{\alpha^4}{2n^4} \left( \frac{n}{l + 1/2} - \frac{3}{4} \right) + ... \right], \label{eq:1.1.6}
\end{equation}
где $\alpha \equiv e^2 / (4 \pi \hbar c)$ --- \textquotedblleftпостоянная тонкой структуры\textquotedblright\ , численное значение которой составляет приблизительно $1/127$, $n$ --- положительное целое число, а $l$ --- орбитальный угловой момент в единицах $\hbar$, принимающий целочисленные значения в интервале $0 \le l \le n-1$.
\end{document}
