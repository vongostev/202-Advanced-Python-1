\documentclass[a4paper,12pt]{article}
\usepackage[utf8]{inputenc}
\usepackage[T2A]{fontenc}
\usepackage{geometry}
\geometry{papersize={21cm,55.4cm},left=29mm,right=29mm,top = 3 cm, bottom = 25cm }
\usepackage{float}
\usepackage{amsmath}

\parindent = 0pt

\setcounter{page}{19}
\usepackage{fancyhdr}
\fancyhead[R]{\thepage}

\fancyhead[C]{\textit{\textsection 1.1 Релятивисткая волновая механика}}
\fancyfoot[L]{2*}
\fancyfoot[C]{}
\begin{document}
\pagestyle{fancy}
\large
опубликовать свое релятивистское волновое уравнение, оно уже было независимо переоткрыто Оскаром Клейном [7] и Вальтером Гордо\-ном [8]. По этой причине релятивистский вариант называется "урав\-нением Клейна-Гордона".
\setlength{\parindent}{3ex}

Шредингер вывел свое релятивистское волновое уравнение, заме\-тив, что гамильтониан $H$ и импульс $\textbf{p}$ ``электрона Лоренца'' с массой $m$ и зарядом $e$, находящегося во внешнем векторном потенциале $\textbf{A}$ и ку\-лоновском потенциале $\phi$, связаны следующим соотношением\footnote{\normalsize) Это соотношение лоренц-инвариантно, поскольку величины $\textbf{A}$ и $\phi$ при преобразованиях Лоренца изменяются точно так же, как $c\textbf{p}$ и $E$ соот\-ветственно. Гамильтониан $H$ и импульс $\textbf{p}$ Шредингер представлял в виде частных производных действия, однако это неважно для нашего рассмот\-рения.}):
\parindent = 0pt
$$0 = (H+e\phi )^2-c^2(\textbf{p}+e\textbf{A} /c)^2-m^2c^4. \eqno(1.1.2)$$
Соотношения де Бройля (1.1.1) для \textit{cвободной} частицы, представлен\-ной плоской волной $\exp \{ 2 \pi i( \boldsymbol {\kappa} \cdot \textbf{x} -vt) \} $, можно получить, если про\-извести отождествление
$$\textbf{p} = h\textbf{k}\rightarrow -i\hbar\nabla,\;\;\;\;\;\; E=h\nu \rightarrow i\hbar \frac{\partial}{\partial t},\eqno(1.1.3)$$
где $\hbar$ — удобное обозначение (введенное Дираком) для $h/2 \pi$. Исходя из чисто формальной аналогии Шредингер предположил, что элек\-трон во внешних полях $\textbf{A}$, $\phi$ должен описываться волновой функ\-цией $\psi(\textbf{x},t)$, удовлетворяющей уравнению, получаемому при помощи той же самой замены в (1.1.2):
$$0 = \biggl[\biggl(i\hbar \frac{\partial}{\partial t}+e\phi \biggr)^2 - c^2\biggl(-i\hbar \nabla +\frac{e \textbf{A}}{c}\biggr)^2-m^2c^4\biggr]\psi (\textbf {x},t).\eqno(1.1.4)$$
В частности, для стационарных состояний в атоме водорода справед\-ливы равенства $\textbf{A} = 0$ и $\psi = e/(4\pi r)$. Кроме того, в этом случае $\psi$ зависит от времени $t$ экспоненциально: $\exp(-iEt/\hbar)$. Поэтому (1.1.4) сводится к уравнению
$$0 = \biggl[\biggl(E+\frac{e^2}{4\pi r}\biggr)^2-c^2\hbar^2\nabla^2-m^2c^4\biggr]\psi (\textbf{x}) \eqno(1.1.5)$$
Решения уравнения (1.1.4) с наложенными на них разумными гранич\-ными условиями, определяют уровни энергии [9]
$$E = mc^2\biggl[1-\frac{\alpha^2}{2n^2}-\frac{\alpha^4}{2n^4}\biggl(\frac{n}{l+1/2}-\frac{3}{4}\biggl)+\ldots \biggr], \eqno(1.1.6)$$
где $\alpha\equiv /(4\pi\hbar c)$ -- ``постоянная тонкой структуры'', численное зна\-чение которой составляет приблизительно $1/137$, $n$ -- положительное целое число, а $l$ -- орбитальный угловой момент в единицах $\hbar$, прини\-мающий целочисленные значения в интервале $0\le l\le n-1$. Наличие
\end{document}